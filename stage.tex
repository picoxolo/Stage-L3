\documentclass{article} %[12pt]
\usepackage[utf8]{inputenc}
\usepackage[T1]{fontenc}
\usepackage[francais]{babel}
\usepackage{amssymb}
\usepackage{amsmath}
\usepackage{xcolor}
\title{\textbf{Stage L3}}
\author{Luc GUILLOT - Astrid LACOTTE}
\setlength{\parindent}{0cm}


\begin{document}
\maketitle


\begin{section}{Définitions}

U suit une loi gaussienne de paramètres $N(0, \Gamma)$.

$U = k \ast W$ avec $k \in M_{M,N}(\mathbb{R}), W \in M_{M,N}(\mathbb{R})$


On assimile $\Omega$ à $\Omega = \frac{\mathbb{Z}}{M\mathbb{Z}} \times \frac{\mathbb{Z}}{N\mathbb{Z}}$.
\\
$\mathcal{F}(s)(x) = \sum_{ y \in \Omega}{s(y) \cdot e^{-2i \pi \langle x,y \rangle}}$ avec 
$\langle x,y \rangle =  \frac{ x_1 y_1}{M} + \frac{ x_2 y_2}{N}$ pour $x = (x_1,x_2)$ et $y = (y_1,y_2)$
\\
$A(k)(x) = \sum_{ y \in \Omega}{k(y)\bar{k(x+y)}}$ \
$A(k) = \mathcal{F}^{-1}( {\lvert \mathcal{F}(k) \rvert }^2)$
\\
$f \ast g (x) = \sum_{ y \in \Omega}{f(y)g(y-x)}$
\\
$\Gamma = \sigma^{2} \cdot A(k) $
\\
$k_0 = \mathcal{F}^{-1}( \sqrt{(\lvert \mathcal{F}(A(k) \rvert) }) = \mathcal{F}^{-1}( \sqrt{( \mathcal{F}(\frac{\Gamma}{\sigma^2}  }))$
\\
Petite propriété:
\\
$\mathcal{F}(A(s))(x) = \mathcal{F}(s)^2(x)$ 
\\
Rappel : Il y a équivalence entre les trois points suivants:
$k_1 \ast W = k_2 \ast W$
\\
$A(k_1) = A(k_2)$
\\
$\lvert \mathcal{F}(k_1) \rvert = \lvert \mathcal{F}(k_2) \rvert$

A partir d'une image $u = k \ast W$,$\mu = 0$, $\sigma = 1$, on définit:

$\Gamma_{emp} = \tilde{\Gamma} = \frac{A(u)}{MN}$ \
$k_{0,emp} = \tilde{k_0}= \mathcal{F}^{-1}( \sqrt{(\lvert \mathcal{F}(\tilde{\Gamma} \rvert) })$



\end{section}

\begin{section}{Premiers résultats}

$\mathbb{E}[\tilde{\Gamma}](x) = \tilde{\Gamma}(x) $\\
$\mathbb{V}[\tilde{\Gamma}](x) = \frac{2 {\left\| \Gamma \right\|}^2_2}{(MN)^2}$
\\
\\
$\mathbb{E}[\mathcal{F}(\tilde{k_0})^2](x) = \mathcal{F}({k_0})^2(x)$
\\
$\mathbb{V}[\mathcal{F}(\tilde{k_0})^2](x) = \mathcal{F}({k_0})^4(x)$
\\
\\
$\mathbb{E}[\mathcal{F}(\tilde{k_0})^2](x) = \mathbb{E}[\frac{\mathcal{F}(\tilde{\Gamma})}{\sigma^2}](x) = \frac{1}{\sigma^2}\cdot \mathbb{E}[\sum_{ y \in \Omega}{\tilde\Gamma(y) \cdot e^{-2i \pi \langle x,y \rangle}}] = \frac{1}{\sigma^2}\cdot \sum_{ y \in \Omega}{\mathbb{E}[\tilde\Gamma(y)] \cdot e^{-2i \pi \langle x,y \rangle}} = 
\frac{1}{\sigma^2}\cdot \sum_{ y \in \Omega}{\Gamma(y) \cdot e^{-2i \pi \langle x,y \rangle}} =\frac{1}{\sigma^2}\cdot \mathcal{F}(\Gamma)(x) =\mathcal{F}(A(k_0))(x) = \mathcal{F}(k_0)^2(x) $
\\
\\
$\mathbb{V}[\mathcal{F}(\tilde{k_0})^2](x) = \mathbb{E}[\mathcal{F}(\tilde{k_0})^4](x) - \mathbb{E}[\mathcal{F}(\tilde{k_0})^2]^2(x)$.
\\
D'une part, $\mathbb{E}[\mathcal{F}(\tilde{k_0})^4](x) = $


\end{section}

\end{document}
